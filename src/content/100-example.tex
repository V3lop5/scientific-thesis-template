\blinddocument

\chapter{Mein Kapitel}\label{ch:example1}
\section{Funny section}\label{sec:funny}
\Blindtext[5][3]

Und noch ein toller Satz zum Abschied\footnote{Abschied kommt meistens zum Schluss.}.

\Blindtext

\chapter{Zweites Kapitel}\label{ch:example2}
Dieser Satz stammt von einer Quelle.\cite{github}
Ein \Gls{framework} ist ein tolles Ding.
Und \gls{fhac} meint die \gls{fhac}-Klasse.

Und \Gls{api} ist eine super \Gls{api}.
\glspl{api} sind eine super!

\chapter{Überschriftentest}\label{ch:example3}
\section{Abschnitt}\label{sec:example3}
\subsection{Unterabschnitt}\label{subsec:example3-1}
\subsubsection{Unterunterabschnitt}\label{subsubsec:example3-1-1}
\subsubsection{Unterunterabschnitt}\label{subsubsec:example3-1-2}
\subsection{Unterabschnitt}\label{subsec:example3-2}
\subsubsection{Unterunterabschnitt}\label{subsubsec:example3-2-1}
\subsubsection{Unterunterabschnitt}\label{subsubsec:example3-2-2}
\section{Abschnitt}\label{sec:example4}
Funny\footnote{Funny ist das englische Wort für lustig.} yooo.

\chapter{Codebeispiele}\label{ch:example5}
In Kotlin kann eine String Methode einfach erstellt werden:

\kotlin{fun Any.asString() = this.toString() + "Hallo dies ist ein Test"}
Und weiter gehts.

Und noch ein Kotlinfile

\kotlinfile{Funny.kt}{Beispiel für irgendwas.}

Hello?

\landscapeimage{d3-cloud.png}{Funny image}