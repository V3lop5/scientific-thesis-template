\chapter{Pakete}\label{ch:packages}

In diesem Template sind verschiedene Pakete eingebunden, die für das Schreiben von wissenschaftlichen Arbeiten im Informatik-Bereich nützlich sind.
In den nachfolgenden Abschnitten wird auf die einzelnen Pakete und deren nutzen eingegangen.
Sämtliche Pakete werden in der \directory{src/config/thesis.sty} eingebunden.

\section{Bibliographie}\label{sec:bibliography}
- Biblatex
- Biber
- Back Reference
- Farbliche Hervorhebung
- Definition über references.bib

\section{Glossare \& Abkürzungen}\label{sec:glossaries}
- Glossaries
- Abkürzungen
- Definition über die zwei Dateien


\section{Aufzählungen}\label{sec:lists}

\section{Figuren}\label{sec:figures}

\subsection{Tabellen}\label{subsec:tables}


\subsection{Bilder}\label{subsec:images}
- Landscape
- subfig

Ein eingefügtes Bild wird von \LaTeX bestmöglich im Text eingefügt.


\subsection{Quellcode}\label{subsec:code}
- Listings
- Verschiedene Befehle für die verschiedensten Sprachen

\section{Mathe}\label{sec:math}
- icoma
- amsmath

Mehrere Formeln können in einer Umgebung stehen, die mit \verb|\begin{align}| und \verb|\end{align}| eingeleitet und beendet wird.
Die Formeln werden dabei automatisch nummeriert.
Durch das Schlüsselwort \verb|\label| kann eine Formel mit einem Label versehen werden, das dann in der Fußnote der Formel angezeigt wird.
Zusätzlich können die Formeln zentriert werden, indem die Umgebung mit \verb|\begin{align*}| und \verb|\end{align*}| eingeleitet und beendet wird.

\section{Zeichen \& Symbole}\label{sec:symbols}
- charter
- wasysym
- menukeys
- directory

\section{Farben}\label{sec:colors}
- xcolor

\section{TODO-Notizen}\label{sec:todo}
\inProgress

Während des Schreibflusses können nicht alle Sachen direkt umgesetzt werden.
Daher ist es sinnvoll, sich Notizen zu machen, die später bearbeitet werden können.

Mit dem Befehl \verb|\todo| können solche Notizen erstellt werden \todo{Satz vervollständigen}

Ein weiterer Befehl ist der \verb|\inProgress| Befehl, welcher einen kleinen Marker anzeigt, dass das Kapitel noch nicht fertig ist.
Die Korrekturleser wissen dadurch, dass sie dieses Kapitel oder Abschnitt erstmal überspringen können.

\section{Das PDF}\label{sec:pdf}
Damit der Leser Spaß am Lesen hat, wurden in diesem Template Pakte eingebunden, die das PDF funktional ansprechender machen.

Dazu zählt zum Beispiel das Paket \verb|hyperref|, das die Verlinkung zwischen den Kapiteln und den Fußnoten ermöglicht.
Außerdem wird die PDF-Gliederung automatisch erstellt.

Das Paket \verb|cmap| ermöglicht es, Texte in mit der Maus zu markieren und diese dann zu kopieren.

In \labelref{sec:bibliography} wurde bereits darauf eingegangen, dass Quellenverweise angeklickt werden können, um zu der entsprechenden Quelle im Literaturverzeichnis zu springen.
Durch die Ergänzung des Hinweises \textit{Referenziert auf der Seite 42} kann der Leser zu der ursprünglichen Textpassage zurückkehren.

\section{Die Basics}\label{sec:basics}
