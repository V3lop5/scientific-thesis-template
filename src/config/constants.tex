%! suppress = DuplicateDefinition
%! Author = Paul
%! Date = 04.02.2022

% Alle möglichen Konstanten, die in diesem Dokument verwendet werden.
% TODO Diese Konstanten müssen aktualisiert werden!

\newcommand{\IsTrue}{1}
\newcommand{\IsFalse}{0}

% Allgemeine Einstellungen

% Handelt es sich um die Druckausgabe? (IsTrue/IsFalse)
\newcommand{\Druckversion}{\IsFalse}

\newcommand{\Subject}{Masterarbeit}
\newcommand{\Titel}{Titel meiner Arbeit}

\newcommand{\Autor}{Max Muster}
\newcommand{\AutorStrasse}{Hauptstr. 1}
\newcommand{\AutorPLZ}{50667}
\newcommand{\AutorOrt}{Köln}

\newcommand{\Matrikelnummer}{1337}
\newcommand{\Abschluss}{Master of Science (M. Sc.)}
\newcommand{\Studiengang}{Energiewirtschaft und Informatik}
\newcommand{\Schwerpunkt}{Wirtschaftsinformatik}

\newcommand{\Erstpruefer}{Prof. Dr.-Ing. Max Muster}
\newcommand{\Zweitpruefer}{Max Muster, M. Sc.}
\newcommand{\VorgelegtAm}{\today}

\newcommand{\InFirma}{\IsTrue}
\newcommand{\Firma}{rhenag Rheinische Energie AG}
\newcommand{\FirmaStrasse}{Bayenthalgürtel 9}
\newcommand{\FirmaPLZ}{50968}
\newcommand{\FirmaOrt}{Köln}

\newcommand{\ArbeitKopfzeile}{TODO Kopfzeile}
\newcommand{\ArbeitThema}{TODO Thema der Arbeit für Kopfzeile}

% Spezialisierte Konstanten (aus obigen erstellt)

% PDF Metadaten
\newcommand{\PDFAutor}{\Autor}
\newcommand{\PDFTitle}{\Titel}
\newcommand{\PDFSubject}{\Subject}

% Titelseite
\newcommand{\TitlepageTitle}{\Titel} % TODO ggf. Umbrüche einfügen
\newcommand{\TitlepageSubject}{\Subject}
\newcommand{\TitlepageSubjectDetail}{
    {zur Erlangung des akademischen Grades \Abschluss}\\
    {im Studiengang \Studiengang}\\
    {mit Schwerpunkt \Schwerpunkt}
}
\newcommand{\TitlepageAuthor}{\Autor}
\newcommand{\TitlepageAuthorStreet}{\AutorStrasse}
\newcommand{\TitlepageAuthorCity}{\AutorPLZ\space\AutorOrt}
\newcommand{\TitlepageAuthorMatrikel}{\Matrikelnummer}

\newcommand{\TitlepageShowFirma}{\InFirma}
\newcommand{\TitlepageFirma}{\FirmaStrasse, \FirmaPLZ\space\FirmaOrt}

% Eigenständigkeitserklärung
\newcommand{\IndependenceCity}{\AutorOrt}
\newcommand{\IndependenceDate}{\VorgelegtAm}

\newcommand{\IndependencePrintSignature}{\IsTrue}
\if\Druckversion\IsTrue
    % In Druckversion keine Unterschrift ausgeben
    \renewcommand{\IndependencePrintSignature}{\IsFalse}
\fi

% Sperrvermerk
\newcommand{\BlockingNoticeShow}{\InFirma} % Sperrvermerk anzeigen? \IsTrue oder \IsFalse
\newcommand{\BlockingNoticeTitle}{\Titel}
\newcommand{\BlockingNoticeFirma}{
\Firma \\
\FirmaStrasse \\
\FirmaPLZ\space\FirmaOrt
}
